\chapter{Universal IO and Connecting PRU to the Outside World}

This project did not require a custom ``Device Tree Overlay''.  Instead, the ``Universal IO'' driver was used along with a simple shell script.

The Universal IO project is located at this Github project:

\url{https://github.com/cdsteinkuehler/beaglebone-universal-io}

The configuration is as follows:

\begin{verbatim}
config-pin P8.44 pruout
config-pin P9.31 pruout
config-pin P9.27 pruout
config-pin P9.29 pruout
config-pin P9.28 pruin
config-pin P9.30 pruout
\end{verbatim}

The above can also be put into a file, for example ``pru-config'':

\begin{verbatim}
P8.44 pruout
P9.31 pruout
P9.27 pruout
P9.29 pruout
P9.28 pruin
P9.30 pruout
\end{verbatim}

The command to load the above would be:

config-pin -f pru-config

Note that another step required is create the \$SLOTS environment variable
and also to set on of the Universal-IO device trees as follows:

\begin{verbatim}
export SLOTS=/sys/devices/platform/bone_capemgr
echo univ-emmc > $SLOTS/slots 
\end{verbatim}


\subsection{The PRU GPIO Spreadsheet}

Use ``git clone'' to download this repository:

\url{https://github.com/selsinork/beaglebone-black-pinmux}

The spreadsheet file contained in this repository is pinmux.ods.
The LibreOffice suite has a spreadsheet application which will read
this file.

This spreadsheet is extremely useful when configuring the PRU or
other functions to the Beaglebone pin multiplexer.



