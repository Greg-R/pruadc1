%    Documentation for PRU ADC Project
%    Copyright (C) 2016  Gregory Raven
%
%    This program is free software: you can redistribute it and/or modify
%    it under the terms of the GNU General Public License as published by
%    the Free Software Foundation, either version 3 of the License, or
%    (at your option) any later version.
%
%    This program is distributed in the hope that it will be useful,
%    but WITHOUT ANY WARRANTY; without even the implied warranty of
%    MERCHANTABILITY or FITNESS FOR A PARTICULAR PURPOSE.  See the
%    GNU General Public License for more details.
%
%    You should have received a copy of the GNU General Public License
%    along with this program.  If not, see <http://www.gnu.org/licenses/>.

\chapter{Introduction}

This is the documentation for a small embedded GNU/Linux project utilizing the RemoteProc and RPMsg framework in the Beaglebone Green (BBG) development board.  The project repository is located here:

\url{https://github.com/Greg-R/pruadc1}

The inspiration for this project came from the superb book ``Exploring Beaglebone'' by Derek Molloy.  Professor Molloy's project utilized assembly code and the UIO driver.  The hardware used in this project is essentially a copy of the Molloy design.

Recent developments in the Texas Instruments PRU support include the RemoteProc and Remote Messaging frameworks, as well as an extensively documented C compiler and much additional supporting documentation.  This project utilizes these frameworks and is entirely dependent upon C code in both the PRU and GNU/Linux user space.  The detailed examples provided by TI in the ``PRU Support Package'' were invaluable in developing this project:

\url{https://git.ti.com/pru-software-support-package}

A listing of additional resources is found in the Resources chapter.

An MCP3008 Analog-to-Digital Converter (ADC) IC is connected to the BBG GPIO pins which are in turn connected to the PRU using the ``Universal IO'' kernel driver which is deployed by default to the current distributions of Debian on the Beaglebones.

The Digilent Analog Discovery 2 was used to analyze and debug the SPI bus between the ADC and the BBG/PRU.  The Analog Discovery 2 was an indispensable tool during the development process.

\section{Project Goals}

This project does not solve a specific practical problem.  It is a laboratory experiment.

The primary goal is to learn to program the PRUs, and to control and communicate with them from the operating system's ``user space''.  Here is a list of the project goals and sub-goals:

\begin{itemize}
	\item Write C code for the PRUs which implement a SPI bus.
	\item Build a ``proto-cape'' with an ADC.
	\item Use the ``Universal IO'' to configure the PRU interface to the GPIO pins.
	\item Write a user-space C program which communicates with the PRUs via character devices.
	\item Use the Analog Discovery 2 as a logic analyzer to debug the PRU SPI C code.
	\item Stream the digitized audio data from the ADC to the ``Advanced Linux Sound Architecture'' in real time.  Play the audio to a speaker.
	\item Document and publish the project to Github.
\end{itemize}

\section{Limitations}

The BeagleBone's Sitara ``System On Chip'' is quite powerful, and this power is probably masking several inefficiencies in the project's design.  All of the testing and debugging was done with a bare-bones Debian distribution with no other significant processes running.

The speaker audio does not have noticeable distortion or glitches when listening with the speaker.  The audio was not examined with a spectrum or distortion analyzer.  It may not be the best quality audio.

The audio sample rate was limited to 8 kHz, which is the lowest rate accepted by ALSA.  A follow-up investigation will see if the sample rate can be increased.  It is not known what aspect of the system will begin to break down as sample rate is increased.

The sample rate had to be ``tuned'' to prevent ``buffer underruns'' reported by the ALSA system.  An approach to make the real-time data stream robust is not known by the author and needs further investigation.


