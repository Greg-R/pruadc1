%    Documentation for PRU ADC Project
%    Copyright (C) 2016  Gregory Raven
%
%    This program is free software: you can redistribute it and/or modify
%    it under the terms of the GNU General Public License as published by
%    the Free Software Foundation, either version 3 of the License, or
%    (at your option) any later version.
%
%    This program is distributed in the hope that it will be useful,
%    but WITHOUT ANY WARRANTY; without even the implied warranty of
%    MERCHANTABILITY or FITNESS FOR A PARTICULAR PURPOSE.  See the
%    GNU General Public License for more details.
%
%    You should have received a copy of the GNU General Public License
%    along with this program.  If not, see <http://www.gnu.org/licenses/>.

\chapter{Introduction}

This is the documentation for a small embedded GNU/Linux project utilizing the RemoteProc and RPMsg framework in the Beaglebone Green (BBG) development board.  The project repository is located here:

\url{https://github.com/Greg-R/pruadc1}

The inspiration for this project came from the superb book ``Exploring Beaglebone'' by Derek Molloy.  Professor Molloy's project utilized assembly code and the UIO driver.  The hardware used in this project is essentially a copy of the Molloy design.

Recent developments in the Texas Instruments PRU support include the RemoteProc and Remote Messaging frameworks, as well as an extensively documented C compiler and much additional supporting documentation.  This project utilizes these frameworks and is entirely dependent upon C code in both the PRU and GNU/Linux user space.  The detailed examples provided by TI in the ``PRU Support Package'' were invaluable in developing this project:

\url{https://git.ti.com/pru-software-support-package}

A listing of additional resources is found in the Resources chapter.

An MCP3008 Analog-to-Digital Converter (ADC) IC is connected to the BBG GPIO pins which are in turn connected to the PRU using the ``Universal IO'' kernel driver which is deployed by default to the current distributions of Debian on the Beaglebones.

The Digilent Analog Discovery 2 was used to analyze and debug the SPI bus between the ADC and the BBG/PRU.  The Analog Discovery 2 was an indispensable tool during the development process.
