%    Documentation for PRU ADC Project
%    Copyright (C) 2016  Gregory Raven
%
%    This program is free software: you can redistribute it and/or modify
%    it under the terms of the GNU General Public License as published by
%    the Free Software Foundation, either version 3 of the License, or
%    (at your option) any later version.
%
%    This program is distributed in the hope that it will be useful,
%    but WITHOUT ANY WARRANTY; without even the implied warranty of
%    MERCHANTABILITY or FITNESS FOR A PARTICULAR PURPOSE.  See the
%    GNU General Public License for more details.
%
%    You should have received a copy of the GNU General Public License
%    along with this program.  If not, see <http://www.gnu.org/licenses/>.

\chapter{Incorporating Advanced Linux Sound Architecture or ``ALSA''}

The system generates a stream of audio data samples at a rate of 8 kHz.  This data could be stored to memory and later manipulated.  Derek Molloy's project used ``GNU Plot'' to graph data captured by the ADC-PRU system and stored to the Beaglebone's memory (RAM).

One of the primary goals of this project was to investigate real-time data "streaming".  So rather than a static capture, it was decided that the data stream would be somehow extracted from the system and ``played'' to an analog speaker.

The primary sound system in GNU/Linux is called the ``Advanced Linux Sound Architecture''.  This system is very mature and flexible, and also it has a very complex Applications Programming Interface (API).

Fortunately the ALSA system includes command line utilities which made meeting the project goals very simple!

\subsection{Pulse Code Modulation}

What kind of data is delivered by the PRU as it reads the ADC via SPI bus?

The ADC samples data with a ``resolution'' of 10 bits.  What this means is that the analog input of the ADC is ``quantized'' into $2^{10}$ or 1024 slices.

The ADC is capable of handling an input range from a little bit above 0 Volts to a little bit below 3.3 Volts.  So a good approximation is a 3.0 Volts range.  Splitting this into 1024 slices yields:

\[ \frac{\text{3.0 Volts}}{1024}  = \text{2.9 millivolts per step} \]
