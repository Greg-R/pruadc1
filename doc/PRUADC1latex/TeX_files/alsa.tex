%    Documentation for PRU ADC Project
%    Copyright (C) 2016  Gregory Raven
%
%    This program is free software: you can redistribute it and/or modify
%    it under the terms of the GNU General Public License as published by
%    the Free Software Foundation, either version 3 of the License, or
%    (at your option) any later version.
%
%    This program is distributed in the hope that it will be useful,
%    but WITHOUT ANY WARRANTY; without even the implied warranty of
%    MERCHANTABILITY or FITNESS FOR A PARTICULAR PURPOSE.  See the
%    GNU General Public License for more details.
%
%    You should have received a copy of the GNU General Public License
%    along with this program.  If not, see <http://www.gnu.org/licenses/>.

\chapter{Incorporating Advanced Linux Sound Architecture or ``ALSA''}

The system generates a stream of audio data samples at a rate of 8 kHz.  This data could be stored to memory and later manipulated.  Derek Molloy's project used ``GNU Plot'' to graph data captured by the ADC-PRU system and stored to the Beaglebone's memory (RAM).

One of the primary goals of this project was to investigate real-time data "streaming".  So rather than a static capture, it was decided that the data stream would be somehow extracted from the system and ``played'' to an analog speaker.

The primary sound system in GNU/Linux is called the ``Advanced Linux Sound Architecture''.  This system is very mature and flexible, and also it has a very complex Applications Programming Interface (API).

Fortunately the ALSA system includes command line utilities which made meeting the project goals very simple!

\section{Pulse Code Modulation}

What kind of data is delivered by the PRU as it reads the ADC via SPI bus?

The ADC samples data with a ``resolution'' of 10 bits.  What this means is that the analog input of the ADC is ``quantized'' into $2^{10}$ or 1024 slices.

The ADC is capable of handling an input range from a little bit above 0 Volts to a little bit below 3.3 Volts.  So a good approximation is a 3.0 Volts range.  Splitting this into 1024 slices yields:

\[ \frac{\text{3.0 Volts}}{1024}  = \text{2.9 millivolts per step} \]

So the natural data type which is retrieved from the ADC is an ``unsigned integer'' in the range of 0 to 1023.  Each numerical step in this range represents an absolute voltage at the input of the ADC in increments of 2.9 millivolts.

``Pulse Code Modulation'' means a waveform is represented as a series of numbers, which are integers in this case.  The ALSA system is capable of handling a stream of Pulse Code Modulated integers, and the data flowing from the PRU0 to user space is perfectly compatible.

ALSA handles several formats of PCM encoded data, and after numerous trials, it was found that the format ``S16\_LE'' was the easiest to deal with.  The code S16\_LE translates to ``16 bit signed integers, little endian''.

The native data from the ADC is unsigned.  However, it is an easy matter to shift the data and cast it to a 16 bit signed integer:

\begin{verbatim}
    payload[dataCounter] = 50 * ((int16_t)data - 512);
\end{verbatim}

The data is shifted by 1024/2 and then cast to a 16 bit signed integer.  The multiplication fact of 50 expands the absolute range of the integers in order to better utilize the 16 bit dynamic range of the ALSA PCM system.

It is possible that this transformation should be performed in user space rather than use the limited resources of the PRU.  That will be the subject of a future trial.

\section{ALSA's aplay Utility Program}

aplay is a command line tool which is a convenient wrapper for the ALSA soundcard driver.

In a shell, type

\begin{verbatim}
man aplay
\end{verbatim}

to see the various options available for aplay.

The user-space program uses aplay with options as follows:

\begin{verbatim}
aplay --format=S16_LE -Dplughw:1,0 --rate=8000 soundfifo
\end{verbatim}

This example command plays file soundfifo using a PCM format of signed 16-bit integers on soundcard 1, device 0, at a rate of 8000 samples per second.

Devices can be discovered via the -l option:

\begin{verbatim}
debian@arm:~$ aplay -l
**** List of PLAYBACK Hardware Devices ****
card 1: Device [C-Media USB Audio Device], device 0: USB Audio [USB Audio]
  Subdevices: 1/1
  Subdevice #0: subdevice #0
\end{verbatim}

The above output was from a BeagleBone Green with a USB audio codec attached.






